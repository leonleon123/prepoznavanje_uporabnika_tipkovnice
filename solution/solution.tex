\documentclass[12pt]{article}

\usepackage[utf8]{inputenc}
\usepackage[T1]{fontenc}
\usepackage{amsmath}
\usepackage{bm}
\usepackage[a4paper, top=20mm]{geometry}
\usepackage{pdfpages}
\usepackage{listings}

\lstset{
  basicstyle=\ttfamily\small, 
  language=Matlab,
  keywordstyle=\color{blue}
}

\begin{document}
\title{\textbf{Prepoznavanje uporabnika tipkovnice}}
\author{
  Leon Modic\\
  \and
  Matej Miočič\\
  \and
  Andraž Rozman\\
}
\maketitle

\section{Opis problema}
Različni ljudje razvijemo različne stile tipkanja. Naloga je, napišete program,
ki bo na podlagi količin, ki jih lahko izmerimo pri tipkanju (kot je časovni
zamik med posameznimi pari znakov), prepoznal uporabnika.
\section{Opis matematičnega modela}
Izbrali smo 47 tipk na tipkovnici in za vsakega uporabnika smo generirali več učnih
primerov matrik povprečnih časov velikosti $47 \times 47$
\[
  averageMatrix^{47 \times 47}=
  \begin{bmatrix}
    0.1301 & 0.0 & \dots & 0.0 & 0.0\\
    0.0 & 0.0 & \dots & 0.0 & 0.0\\
    \vdots & & & \vdots\\
    0.0 & 0.122 & \dots & 0.0 & 0.1\\
    0.0 & 0.0 & \dots & 0.735 & 0.0\\
  \end{bmatrix}
\]

Za vsakega izmed uporabnikov znotraj mape \texttt{data/} se generira $A_i$ matrika
(torej $A_{aljaz}$, $A_{andraz}$, $A_{leon}$, $A_{Matej}$), ki je sestavljena iz matrix povprečnih
časov za vsa merjenja določenega uporabnika.
\[
  A_i = \begin{bmatrix}
    p_1 & p_2 & \dots & p_m
  \end{bmatrix}
\]
Kjer so $p_i$ $2209 \times 1$ vektorji sestavljeni iz stolpcev matrike povprečnih časov ($averageMatrix$) zloženi
eden nad drugega. Nato naredimo SVD razcep:
$$A_i=U_iS_iV_i^T$$
$b$ je $2209 \times 1$ vektor sestavljen iz stolpcev matrike povprečnih časov, ki se je generirala za trenutno osebo, kateri želimo
določiti ime.
$$U_iS_iy_i = b \rightarrow y_i = (U_iS_i)^+b$$
$$min(||b-A_ix||) = min(||U_i^Tb-S_iy_i||)$$

\[
  names = \begin{bmatrix}
    "aljaz"\\
    "andraz"\\
    "leon"\\
    "Matej"
  \end{bmatrix},
  norms = \begin{bmatrix}
    1.92\\
    1.2\\
    1.44\\
    1.35
  \end{bmatrix}
\]
\section{Opis programske kode}
Izbrali smo naslednih 47 tipk katere beležimo v matriko povprečnih časov
% \begin{lstlisting}
%   A = svd([1 2 3; 3 4 5]);
%   test = "skdasd";
%   for i=[1 2 3]
%     i
%   end
% \end{lstlisting}

\section{Rezultati in komentarji rezultatov}

\section{Razdelitev dela v skupini}
Večinoma smo vso kodo pisali tako, da je tisti, ki je delil zalson dejansko tipkal,
skupaj pa smo razmišljali kaj na se napiše. 
\section{Reference in dejanksa koda}

\end{document}
